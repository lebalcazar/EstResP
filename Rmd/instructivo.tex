% Options for packages loaded elsewhere
\PassOptionsToPackage{unicode}{hyperref}
\PassOptionsToPackage{hyphens}{url}
\PassOptionsToPackage{dvipsnames,svgnames,x11names}{xcolor}
%
\documentclass[
]{article}
\usepackage{amsmath,amssymb}
\usepackage{lmodern}
\usepackage{iftex}
\ifPDFTeX
  \usepackage[T1]{fontenc}
  \usepackage[utf8]{inputenc}
  \usepackage{textcomp} % provide euro and other symbols
\else % if luatex or xetex
  \usepackage{unicode-math}
  \defaultfontfeatures{Scale=MatchLowercase}
  \defaultfontfeatures[\rmfamily]{Ligatures=TeX,Scale=1}
\fi
% Use upquote if available, for straight quotes in verbatim environments
\IfFileExists{upquote.sty}{\usepackage{upquote}}{}
\IfFileExists{microtype.sty}{% use microtype if available
  \usepackage[]{microtype}
  \UseMicrotypeSet[protrusion]{basicmath} % disable protrusion for tt fonts
}{}
\makeatletter
\@ifundefined{KOMAClassName}{% if non-KOMA class
  \IfFileExists{parskip.sty}{%
    \usepackage{parskip}
  }{% else
    \setlength{\parindent}{0pt}
    \setlength{\parskip}{6pt plus 2pt minus 1pt}}
}{% if KOMA class
  \KOMAoptions{parskip=half}}
\makeatother
\usepackage{xcolor}
\usepackage[margin=1in]{geometry}
\usepackage{color}
\usepackage{fancyvrb}
\newcommand{\VerbBar}{|}
\newcommand{\VERB}{\Verb[commandchars=\\\{\}]}
\DefineVerbatimEnvironment{Highlighting}{Verbatim}{commandchars=\\\{\}}
% Add ',fontsize=\small' for more characters per line
\usepackage{framed}
\definecolor{shadecolor}{RGB}{248,248,248}
\newenvironment{Shaded}{\begin{snugshade}}{\end{snugshade}}
\newcommand{\AlertTok}[1]{\textcolor[rgb]{0.94,0.16,0.16}{#1}}
\newcommand{\AnnotationTok}[1]{\textcolor[rgb]{0.56,0.35,0.01}{\textbf{\textit{#1}}}}
\newcommand{\AttributeTok}[1]{\textcolor[rgb]{0.77,0.63,0.00}{#1}}
\newcommand{\BaseNTok}[1]{\textcolor[rgb]{0.00,0.00,0.81}{#1}}
\newcommand{\BuiltInTok}[1]{#1}
\newcommand{\CharTok}[1]{\textcolor[rgb]{0.31,0.60,0.02}{#1}}
\newcommand{\CommentTok}[1]{\textcolor[rgb]{0.56,0.35,0.01}{\textit{#1}}}
\newcommand{\CommentVarTok}[1]{\textcolor[rgb]{0.56,0.35,0.01}{\textbf{\textit{#1}}}}
\newcommand{\ConstantTok}[1]{\textcolor[rgb]{0.00,0.00,0.00}{#1}}
\newcommand{\ControlFlowTok}[1]{\textcolor[rgb]{0.13,0.29,0.53}{\textbf{#1}}}
\newcommand{\DataTypeTok}[1]{\textcolor[rgb]{0.13,0.29,0.53}{#1}}
\newcommand{\DecValTok}[1]{\textcolor[rgb]{0.00,0.00,0.81}{#1}}
\newcommand{\DocumentationTok}[1]{\textcolor[rgb]{0.56,0.35,0.01}{\textbf{\textit{#1}}}}
\newcommand{\ErrorTok}[1]{\textcolor[rgb]{0.64,0.00,0.00}{\textbf{#1}}}
\newcommand{\ExtensionTok}[1]{#1}
\newcommand{\FloatTok}[1]{\textcolor[rgb]{0.00,0.00,0.81}{#1}}
\newcommand{\FunctionTok}[1]{\textcolor[rgb]{0.00,0.00,0.00}{#1}}
\newcommand{\ImportTok}[1]{#1}
\newcommand{\InformationTok}[1]{\textcolor[rgb]{0.56,0.35,0.01}{\textbf{\textit{#1}}}}
\newcommand{\KeywordTok}[1]{\textcolor[rgb]{0.13,0.29,0.53}{\textbf{#1}}}
\newcommand{\NormalTok}[1]{#1}
\newcommand{\OperatorTok}[1]{\textcolor[rgb]{0.81,0.36,0.00}{\textbf{#1}}}
\newcommand{\OtherTok}[1]{\textcolor[rgb]{0.56,0.35,0.01}{#1}}
\newcommand{\PreprocessorTok}[1]{\textcolor[rgb]{0.56,0.35,0.01}{\textit{#1}}}
\newcommand{\RegionMarkerTok}[1]{#1}
\newcommand{\SpecialCharTok}[1]{\textcolor[rgb]{0.00,0.00,0.00}{#1}}
\newcommand{\SpecialStringTok}[1]{\textcolor[rgb]{0.31,0.60,0.02}{#1}}
\newcommand{\StringTok}[1]{\textcolor[rgb]{0.31,0.60,0.02}{#1}}
\newcommand{\VariableTok}[1]{\textcolor[rgb]{0.00,0.00,0.00}{#1}}
\newcommand{\VerbatimStringTok}[1]{\textcolor[rgb]{0.31,0.60,0.02}{#1}}
\newcommand{\WarningTok}[1]{\textcolor[rgb]{0.56,0.35,0.01}{\textbf{\textit{#1}}}}
\usepackage{graphicx}
\makeatletter
\def\maxwidth{\ifdim\Gin@nat@width>\linewidth\linewidth\else\Gin@nat@width\fi}
\def\maxheight{\ifdim\Gin@nat@height>\textheight\textheight\else\Gin@nat@height\fi}
\makeatother
% Scale images if necessary, so that they will not overflow the page
% margins by default, and it is still possible to overwrite the defaults
% using explicit options in \includegraphics[width, height, ...]{}
\setkeys{Gin}{width=\maxwidth,height=\maxheight,keepaspectratio}
% Set default figure placement to htbp
\makeatletter
\def\fps@figure{htbp}
\makeatother
\setlength{\emergencystretch}{3em} % prevent overfull lines
\providecommand{\tightlist}{%
  \setlength{\itemsep}{0pt}\setlength{\parskip}{0pt}}
\setcounter{secnumdepth}{-\maxdimen} % remove section numbering
\ifLuaTeX
\usepackage[bidi=basic]{babel}
\else
\usepackage[bidi=default]{babel}
\fi
\babelprovide[main,import]{spanish}
% get rid of language-specific shorthands (see #6817):
\let\LanguageShortHands\languageshorthands
\def\languageshorthands#1{}
\ifLuaTeX
  \usepackage{selnolig}  % disable illegal ligatures
\fi
\IfFileExists{bookmark.sty}{\usepackage{bookmark}}{\usepackage{hyperref}}
\IfFileExists{xurl.sty}{\usepackage{xurl}}{} % add URL line breaks if available
\urlstyle{same} % disable monospaced font for URLs
\hypersetup{
  pdftitle={Bioestadística aplicada a la restauración},
  pdfauthor={Luis Balcázar},
  pdflang={es},
  colorlinks=true,
  linkcolor={blue},
  filecolor={Maroon},
  citecolor={blue},
  urlcolor={blue},
  pdfcreator={LaTeX via pandoc}}

\title{Bioestadística aplicada a la restauración}
\usepackage{etoolbox}
\makeatletter
\providecommand{\subtitle}[1]{% add subtitle to \maketitle
  \apptocmd{\@title}{\par {\large #1 \par}}{}{}
}
\makeatother
\subtitle{Instalación de programas}
\author{Luis Balcázar}
\date{Loja, abril 2023}

\begin{document}
\maketitle

En este documento usted encontrará los \emph{links} para descargar los
programas R, Rtools, RStudio, SAGA, y algunos paquetes de R. La
instalación le tomará un tiempo estimado entre 20 y 30 minutos.

Los datos para el curso se encuentran en el siguiente
\href{https://github.com/lebalcazar/R_IITCA.git}{repositorio} (estará
disponible minutos antes del taller). Es recomendable que usted tenga
otro dispositivo aparte de su computador para ver la presentación y
realizar los ejercicios en la máquina que tiene R y RStudio.

Además, descargue los siguientes datos de
\href{https://www.dropbox.com/s/tkkqyz2kaqoh4cj/DEM_Senegal.7z?dl=0}{Senegal}
y
\href{http://www.conabio.gob.mx/informacion/metadata/gis/dipoest00gw.xml?_xsl=/db/metadata/xsl/fgdc_html.xsl\&_indent=no}{México},
para este último elegir SHP1 ``en coordenadas geográficas''.

\hypertarget{instalaciuxf3n-de-r-y-rstudio}{%
\subsection{Instalación de R y
RStudio}\label{instalaciuxf3n-de-r-y-rstudio}}

Ingrese a la página de \textbf{CRAN} (\emph{The Comprehensive \(R\)
Archive Network}) en \href{https://cran.r-project.org/}{este link} y
descargue \textbf{R} para el sistema operativo de su PC (Linux/Window) o
Mac. En una PC Windows seleccione Download R for Windows/base y
descargue la versión reciente 4.x.x. Además, para máquinas con Windows
es necesario descargar \textbf{Rtools} del
\href{https://cran.r-project.org/bin/windows/Rtools/}{siguiente link}.

Para descargar \textbf{RStudio} ingrese en
\href{https://rstudio.com/products/rstudio/}{este link} y seleccione la
pestaña \emph{DOWNLOAD} (en la parte superior derecha) y luego
\emph{RStudio Desktop} en la versión gratuita.

Instale \textbf{R}, luego \textbf{Rtools} y finalmente \textbf{Rstudio},
siga las indicaciones de instalación típica.

\hypertarget{primera-sesiuxf3n-de-r-desde-rstudio}{%
\subsection{Primera sesión de R desde
RStudio}\label{primera-sesiuxf3n-de-r-desde-rstudio}}

Ejecute RStudio y se cargará automáticamente R. Al abrir RStudio por
primera vez se mostrará como en la Figura \ref{fig1}. Por defecto el
entorno de RStudio tiene 4 paneles, los cuales se pueden modificar en la
pestaña \emph{View/Panes}. Además, en la pestaña \emph{Tools} puede
cambiar la configuración: color de fondo, tamaño y tipo de letra, etc.

\begin{figure}
\centering
\includegraphics{rstudio.png}
\caption{Interfaz de RStudio.\label{fig1}}
\end{figure}

En la Figura \ref{fig1}, el panel (1) es el \emph{\textbf{Editor}} de
código (presione las teclas \texttt{Ctrl+Shift+N} para que aparezca), en
el panel (2) se encuentran la \textbf{\emph{Console}} de R y el
\emph{\textbf{Terminal}}, en el panel (3) se observan las pestañas
\emph{\textbf{Environment}}, \emph{\textbf{History}} y otros, y en el
panel (4) se encuentran las opciones \textbf{\emph{Files}},
\emph{\textbf{Packages}}, \emph{\textbf{Plots}} y \textbf{Help}.

\hypertarget{instalaciuxf3n-de-paquetes}{%
\subsection{Instalación de paquetes}\label{instalaciuxf3n-de-paquetes}}

El programa \texttt{base} de R tiene un amplio conjunto de librerías
para hacer análisis de datos y estadístico, así como gráficos de alta
calidad. Sin embargo, existen otros paquetes que son útiles para mejorar
los resultados y reducir las líneas de código.

En la consola de RStudio escriba \texttt{install.packages()}, como se
muestra en las líneas siguientes, para instalar las librerías tidyverse,
tmap, lubridate y raster.

\begin{Shaded}
\begin{Highlighting}[]
\FunctionTok{install.packages}\NormalTok{(}\StringTok{\textquotesingle{}tidyverse\textquotesingle{}}\NormalTok{, }\StringTok{\textquotesingle{}tmap\textquotesingle{}}\NormalTok{, }\StringTok{\textquotesingle{}lubridate\textquotesingle{}}\NormalTok{, }\StringTok{\textquotesingle{}raster\textquotesingle{}}\NormalTok{)}
\end{Highlighting}
\end{Shaded}

Para cargar las librerías en la sesión de R utilice la función
\texttt{library()}. Observe que para cargar los paquetes, no se utiliza
las comillas en los nombres:

\begin{Shaded}
\begin{Highlighting}[]
\CommentTok{\# librerías que se cargarán en la sesión de R}
\FunctionTok{library}\NormalTok{(tidyverse)  }\CommentTok{\# familia de paquetes para el análisis de datos}
\FunctionTok{library}\NormalTok{(tmap)       }\CommentTok{\# creación de mapas temáticos}
\FunctionTok{library}\NormalTok{(lubridate)  }\CommentTok{\# procesamiento de fecha y tiempo}
\FunctionTok{library}\NormalTok{(raster)     }\CommentTok{\# análisis de datos raster}
\end{Highlighting}
\end{Shaded}

Una vez cargada(s) la(s) librería(s) en la sesión de R, abrir la ayuda
de los paquetes para familiarse con ellos, escriba en la consola o
editor: \texttt{?raster}, \texttt{?tmap}, \texttt{help("tmap")}, etc., o
desde el panel \texttt{Help} puede navegar en la ayuda.

Instale estos otros paquetes:
\texttt{"sf",\ "ggmap",\ "mapproj",\ "osmar",\ "tidyverse",\ "RColorBrewer","dplyr",\ "RgoogleMaps",\ "OpenStreetMap",\ "devtools",\ \ "DT",\ "raster",\ "rgdal",\ "rworldxtra",\ \textquotesingle{}grDevices\textquotesingle{},\ \textquotesingle{}ggsn\textquotesingle{},\ \textquotesingle{}tmap\textquotesingle{},\ \textquotesingle{}utils\textquotesingle{},\ \textquotesingle{}viridis\textquotesingle{},\ "datos",\ \textquotesingle{}RColorBrewer\textquotesingle{},\ \textquotesingle{}RSAGA\textquotesingle{},\ \textquotesingle{}cptcity\textquotesingle{}}.
Utilizar \texttt{install.packages()} como se mostró anteriormente.

Si alguna librería no se instala, ejecute las siguientes líneas, y
vuelva a instalar:

\begin{Shaded}
\begin{Highlighting}[]
\FunctionTok{options}\NormalTok{(}\AttributeTok{download.file.method =} \StringTok{\textquotesingle{}libcurl\textquotesingle{}}\NormalTok{) }\CommentTok{\# para conectar con la URL del paquete }
\FunctionTok{install.packages}\NormalTok{(}\StringTok{\textquotesingle{}nombre\_del\_paquete\textquotesingle{}}\NormalTok{, }\AttributeTok{dependences =} \ConstantTok{TRUE}\NormalTok{)}
\end{Highlighting}
\end{Shaded}

Otra opción para instalar paquetes es descargando los archivos binarios
desde la página de CRAN: \url{https://cran.r-project.org/} ver la Figura
\ref{fig2}. Se descargan los archivos en un .zip. Luego en la pestaña
Tools de RStudio ir a la pestaña Tools/Install Packages/Package Archive
File y buscar la carpeta que contiene el archivo .zip.

\begin{figure}
\centering
\includegraphics{lista_paquetes.png}
\caption{Lista de paquetes de R.\label{fig2}}
\end{figure}

En otras ocasiones (\textbf{NO SERÁ NECESARIO EN ESTE TALLER}) los
programas no están en CRAN y se instalan desde un servicio git (github,
gitlab o bitbucket) o Subversion, se utiliza el paquete \emph{remotes} o
\emph{devtools} de la siguiente manera:

\begin{Shaded}
\begin{Highlighting}[]
\FunctionTok{install.packages}\NormalTok{(}\StringTok{"remotes"}\NormalTok{)}
\CommentTok{\# sin cargar las librerías remotes o devtools}
\NormalTok{remotes}\SpecialCharTok{::}\FunctionTok{install\_github}\NormalTok{(}\StringTok{\textquotesingle{}autor/package\_name\textquotesingle{}}\NormalTok{)}

\CommentTok{\# cargando la librería remotes o devtools}
\FunctionTok{library}\NormalTok{(remotes)}
\FunctionTok{install\_gitlab}\NormalTok{(}\StringTok{\textquotesingle{}autor/package\_name\textquotesingle{}}\NormalTok{)}

\CommentTok{\#Una vez instalado, se puede cargar la librería normalmente}
\FunctionTok{library}\NormalTok{(package\_name)}
\end{Highlighting}
\end{Shaded}

\hypertarget{instalar-saga-gis}{%
\subsection{Instalar SAGA GIS}\label{instalar-saga-gis}}

Descargar e instalar SAGA desde el siguiente
\href{https://sourceforge.net/projects/saga-gis/files/}{link}. Elija la
versión 6.3 que ha sido testeada con la versión de RSAGA.

\begin{figure}
\centering
\includegraphics{saga.png}
\caption{Sitio web de SAGA.}
\end{figure}

\end{document}
